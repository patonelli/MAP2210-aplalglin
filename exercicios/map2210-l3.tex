\documentclass[12pt]{article}
\usepackage{amsfonts, amsmath, amssymb}
\usepackage[brazil]{babel}
\usepackage{graphicx}
\usepackage[T1]{fontenc}
\usepackage{lmodern}

\parindent=0pt

 \addtolength{\textheight}{3.5cm}
 \addtolength{\oddsidemargin}{-1cm}
 \addtolength{\evensidemargin}{-1cm}
 \addtolength{\textwidth}{2cm}
 \addtolength{\topmargin}{-2.0cm}
\newcounter{questao}
\newcommand{\quest}{\stepcounter{questao}{\bf \arabic{questao}.\ }}

\begin{document}
\hrule
 {  \sf Aplicações de Álgebra Linear MAP2210 \hfill \fbox{Lista 3 - 2021}}
\hrule

\vspace{0.5cm}

\thispagestyle{empty}
%\fontsize{14}{16}\selectfont

\quest  Achar os autovalores e autovetores da matriz
$$ A = \begin{bmatrix}
  0 & 1 & 0 \\ 0 & 0 & 1 \\ a & -1 & a
\end{bmatrix} $$

\vspace{0.3cm}

\quest  Observa-se o fluxo entre três bairros de uma cidade $A$ $B$ e $C$. O fluxo
anual é estimado assim
\begin{itemize}
  \item metade da população do bairro $A$ muda-se para o bairro $B$
  \item $20\%$ da população do bairro $A$ muda-se para 0 bairro $C$
  \item $30\%$ da população do bairro $B$ muda-se para 0 bairro $C$
  \item Do bairro $C$ $40\%$ vão para $A$ e $20\%$ vão para $B$
\end{itemize}
Monte uma equação de diferenças para a distribuição anual da população dos três bairros, que se mantém 
constante ( = 300.000 habitantes!). E ache para que valores tende a população quando iniciamos
com $100.000$ pessoas em cada bairro.

\vspace{0.3cm}

\quest Ache a exponencial $\exp(A)$ da matriz:
$$ A = \begin{bmatrix}
  3 & 2 & 0 \\ 0 & 3 & 1 \\ 0 & 0 & 3 
\end{bmatrix} $$

\vspace{0.3cm}

\quest Ache a solução da equação diferencial

\begin{gather*}
  \dot{x}_1 = -3x_1 + x_2 -x_3 \\
  \dot{x}_2 = x_1 -3x_2 + x_3 \\
  \dot{x}_3 = -x_1 + x_2 -3x_3 \\
  x_1(0)=1 \quad x_2(0)=0 \quad x_3(0) = 1
\end{gather*}

\vspace{0.3cm}

\quest Verifique se a matriz 
$$ A =\begin{bmatrix}
  4 & 2 \\ -2 & 0 
\end{bmatrix} $$ é diagonalizável e
ache $\exp(tA)$.


\end{document}

%%% Local Variables: 
%%% mode: latex
%%% TeX-master: t
%%% End: 
