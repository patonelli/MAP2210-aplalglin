\documentclass[12pt]{article}
\usepackage{amsfonts, amsmath, amssymb}
\usepackage[brazil]{babel}
\usepackage{graphicx}
\usepackage[T1]{fontenc}
\usepackage{lmodern}

\parindent=0pt

 \addtolength{\textheight}{3.5cm}
 \addtolength{\oddsidemargin}{-1cm}
 \addtolength{\evensidemargin}{-1cm}
 \addtolength{\textwidth}{2cm}
 \addtolength{\topmargin}{-2.0cm}
\newcounter{questao}
\newcommand{\quest}{\stepcounter{questao}{\bf \arabic{questao}.\ }}

\begin{document}
\hrule
 {  \sf Aplicações de Álgebra Linear MAP2210 \hfill \fbox{Lista 1 - 2021}}
\hrule

\vspace{0.5cm}

\thispagestyle{empty}
%\fontsize{14}{16}\selectfont

\quest  Uma aplicação linear $T: \mathbb{R}^3 \to \mathbb{R}^3$ tem a seguinte representação matricial 
na base canônica:
$$ A_{T}=\begin{pmatrix}
  1& 0 & 2 \\ 0 & 1 & 1 \\ -1 & 2 & 0
\end{pmatrix}$$
Se fizermos uma mudança de base só no contra-domínio da aplicação $T$ substituindo a base canônica $\mathbf{e}_1$, 
$\mathbf{e}_2$ e $\mathbf{e}_3$ por $\mathbf{f}_1= \mathbf{e}_1 + \mathbf{e}_2$, $\mathbf{f}_2= \mathbf{e}_2 + \mathbf{e}_3$
e $\mathbf{f}_3= \mathbf{e}_3 - \mathbf{e}_1$, qual é a nova matriz de representação de $T$ com relação a esta base?


\vspace{0.3cm}

\quest  Achar a decomposição $LU$ e $LDV$ da matriz seguinte
$$A = \begin{pmatrix}
  3& 1 &-1\\
  2 & 0 & 0 \\
  1 & -5 & 2 
\end{pmatrix}$$
Neste exercício $D$ é uma matriz diagonal e $V$ uma matriz triangular superior com $1$s na diagonal.

\vspace{0.3cm}

\quest Calcular a inversa das seguintes matrizes
$$L_1 = \begin{pmatrix}
  1 & 0 & 0 & 0 \\
  0 & 1 & 0 & 0 \\
  0 & a & 1 & 0 \\
  0 & b & 0 & 1
\end{pmatrix} \text{ e }
L_2 = \begin{pmatrix}
  1 & 0 & 0 & 0 \\
  a & 1 & 0 & 0 \\
  b & d & 1 & 0 \\
  c & e & f & 1
\end{pmatrix}
$$

\vspace{0.3cm}

\quest Achar $L$ e $U$ da matriz 
$$ A = \begin{pmatrix}
  a & a & a & a \\
  a & b & b & b \\
  a & b & c & c \\
  a & b & c & d 
\end{pmatrix} $$ e dar as condições sobre $a, b, c, d$ para que não haja permutações de linhas.

\vspace{0.3cm}

\quest Mostre que se uma matriz
$$ A= \begin{pmatrix}
  a_{11} & a_{12} & a_{13} \\
  a_{21} & a_{22} & a_{23} \\
  a_{31} & a_{32} & a_{33} \\
\end{pmatrix} $$ tem o determinante do menor
$$\begin{vmatrix}
  a_{11} & a_{12} \\
  a_{21} & a_{22}  \\
\end{vmatrix} =0 $$ então não é possível escrever $A=LU$.


\end{document}

%%% Local Variables: 
%%% mode: latex
%%% TeX-master: t
%%% End: 
