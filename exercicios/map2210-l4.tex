\documentclass[12pt]{article}
\usepackage{amsfonts, amsmath, amssymb}
\usepackage[brazil]{babel}
\usepackage{graphicx}
\usepackage[T1]{fontenc}
\usepackage{lmodern}

\parindent=0pt

 \addtolength{\textheight}{3.5cm}
 \addtolength{\oddsidemargin}{-1cm}
 \addtolength{\evensidemargin}{-1cm}
 \addtolength{\textwidth}{2cm}
 \addtolength{\topmargin}{-2.0cm}
\newcounter{questao}
\newcommand{\quest}{\stepcounter{questao}{\bf \arabic{questao}.\ }}

\begin{document}
\hrule
 {  \sf Aplicações de Álgebra Linear MAP2210 \hfill \fbox{Lista 4 - 2021}}
\hrule

\vspace{0.5cm}

\thispagestyle{empty}
%\fontsize{14}{16}\selectfont

\quest  Encontre a forma de Jordan da matriz 
$$ A= \begin{bmatrix}
  0 & 1 & 0 \\
  -4 & 4 & -4 \\
  0 & 0 & 0
\end{bmatrix} $$

\vspace{0.3cm}

\quest  Uma matriz $A \in \mathbb{R}^{3\times 3}$ tem um autovalor duplo $1$ e um 
autovalor simples $3$. Os autovetores associados a $1$ e $3$ são respectivamente
$$ \vec{v}_1=\begin{bmatrix}
  1 \\ 0 \\ 1
\end{bmatrix}\text{ e } \vec{v}_3 = \begin{bmatrix}
  -1 \\ 0 \\ 1
\end{bmatrix} $$ e um autovetor generalizado 
$$ \vec{v}_2=\begin{bmatrix}
  0 \\ 1 \\ 2
\end{bmatrix}$$
Ache a solução geral da equação
$ \dot{x}= Ax$

\vspace{0.3cm}

\quest Mostre que uma matriz não nula $N$ nilpotente, não pode ser simétrica.

\vspace{0.3cm}

\quest Ache a decomposição $LDL^T$ da matriz
$$A=\begin{bmatrix}
  2 & -1 & 0 \\
  -1 & 2 & 1 \\
  0 & 1 & -2 
\end{bmatrix}$$
Verifique se $A$ é definida positiva.

\vspace{0.3cm}

\quest Mostre que se $A$ e $B$ são matrizes simétricas definidas positivas, então
$A+B$ também é simétrica definida positiva.


\end{document}

%%% Local Variables: 
%%% mode: latex
%%% TeX-master: t
%%% End: 
